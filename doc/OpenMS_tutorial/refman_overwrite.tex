\documentclass[a4paper]{article}
\usepackage{geometry}
\geometry{
  includeheadfoot,
  margin=2.54cm
}
\usepackage{makeidx}
\usepackage{fancyhdr}
\usepackage{graphicx}
\usepackage{multicol}
\usepackage{float}
\usepackage{alltt}
\usepackage{newtxtext, newtxmath}
% newtxtext loads textcomp only in newer versions and with differing options between versions
% @commands need to be enclosed by makeatletter and makeatother
\makeatletter
\@ifpackageloaded{textcomp} 
  {} % do nothing if loaded by the previous package to avoid option clashes
  {\usepackage{textcomp}} % otherwise load the package
\makeatother
\ifx\pdfoutput\undefined
\usepackage[ps2pdf,
            pagebackref=true,
            colorlinks=true,
            linkcolor=blue
           ]{hyperref}
\usepackage{pspicture}
\else
\usepackage[pdftex,
            pagebackref=true,
            colorlinks=true,
            linkcolor=blue
           ]{hyperref}
\fi
\usepackage{doxygen}
\makeindex
\setcounter{tocdepth}{1}
\renewcommand{\footrulewidth}{0.4pt}
\newcommand{\+}{\discretionary{\mbox{\scriptsize$\hookleftarrow$}}{}{}}
\begin{document}

\begin{titlepage}
\vspace*{7cm}
\begin{center}
{\Large OpenMS Tutorial\\[1ex]\large Version: 2.3.0 }\\
\end{center}
\end{titlepage}

\pagenumbering{arabic}

\setcounter{tocdepth}{2}
\tableofcontents
\pagebreak

\section{Introduction}

	\input{tutorial_introduction}
	\pagebreak
	\input{tutorial_structure}
	\pagebreak
	\input{tutorial_buildingOpenMS}
	\pagebreak
	\input{tutorial_terms}


\pagebreak
\section{OpenMS Library}

	\input{tutorial_library}
	\pagebreak
	\input{lib_overview}
	\pagebreak
	\input{lib_kernelclass}
	\pagebreak
	\input{tutorial_kernel}
	\pagebreak
	\input{tutorial_metadata}
	\pagebreak
	\input{tutorial_format}


\pagebreak
\section{Algorithms}

	\input{tutorial_filtering}
	\pagebreak
	\input{tutorial_transformations}
	\pagebreak
	\input{tutorial_mapalignment}
	\pagebreak
	\input{tutorial_featuregrouping}

\pagebreak
\section{Advanced tutorials}

	\input{tutorial_visual}
	\pagebreak
	\input{tutorial_clustering}
	\pagebreak
	\input{tutorial_pip}
	\pagebreak
	\input{tutorial_howto}


\end{document}
