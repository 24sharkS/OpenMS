
\documentclass[10pt]{article}
% \usepackage{amsmath}
\usepackage{listings}
\usepackage{url}
\setlength{\parindent}{0pt}
\usepackage[parfill]{parskip}
% \usepackage{fullpage}
\usepackage[margin=1.5in]{geometry}



\usepackage[
  pdfpagemode=UseOutlines,
  pdfdisplaydoctitle=true,
  colorlinks=true,
  linkcolor=black,
  filecolor=black,
  pagecolor=black,
  urlcolor=black,
  frenchlinks=true
]{hyperref}

\usepackage{tocloft}% http://ctan.org/pkg/tocloft
\setlength{\cftsubsecnumwidth}{3.5em}% Set length of number width in ToC for \subsection

\title{pyOpenMS 2.6}
\author{OpenMS Development Team}
\date{}
\begin{document}
  \maketitle

  % TODO update numbers from time to time
  pyOpenMS is a set of Python bindings of the C++ OpenMS library. It allows to
  access a large number of objects (350+) and functions (3900+) of the C++
  code directly from Python.
  The main functions of the library are explained in the first section of this
  manual. A list of all wrapped functions can be found in the appendix of this
  manual.

  Since all functions in Python directly call C++ and their function signature
  usually corresponds to the one in C++, the OpenMS documentation is for most
  cases the most complete and up-to-date reference also and applies directly
  to pyOpenMS. In this manual, only differences to the existing documentation
  will be highlighted and some general usecases will be explained. The link to
  the documentation of the latest release can be found here:
  \url{http://open-ms.sourceforge.net/documentation/}.

  \tableofcontents

  \pagebreak

  The following section will explain the most important functions of pyOpenMS
  in more detail with full examples of Python code that can be directly
  executed.

\section{File Input/Output}

  pyOpenMS supports file input and output for various formats. These formats
  include \texttt{DTA2D, DTA, EDTA, FeatureXML, Kroenik, MzData, MzIdentML, IdXML, mzML,
  mzXML, PepXML, ProtXML, TraML, XTandemXML}.

\subsection{Common Pattern}

Most file format objects follow the following idiome

\begin{verbatim}
from pyopenms import *
file = FileObject()
exp = MSExperiment()
file.load(filename_in, exp)
# process data
file.store(filename_in, exp)
\end{verbatim}

where \texttt{FileObject} can be any of \texttt{DTA2DFile, DTAFile, mzXMLFile,
mzMLFile \ldots } (note that the above code will not work directly since
you will need to use a specific file implementation, see below).  

For \texttt{EDTAFile}, you must load and store a \texttt{ConsensusMap} instead
of an \texttt{MSExperiment}.
for \texttt{FeatureXMLFile}, you must load and store a \texttt{FeatureMap} instead
of an \texttt{MSExperiment}.


\subsection{IdXML}
\begin{verbatim}
from pyopenms import *
id_file = IdXMLFile()
filename_in = "input.IdXML"
filename_out = "output.IdXML"
protein_ids = []
peptide_ids = []
id_file.load(filename_in, protein_ids, peptide_ids)
# process
id_file.store(filename_out, protein_ids, peptide_ids)
\end{verbatim}

see also Section~\ref{IdXMLFile}.

\subsection{PepXML}
\begin{verbatim}
from pyopenms import *
id_file = PepXMLFile()
filename_in = "input.pep.xml"
filename_out = "output.pep.xml"
protein_ids = []
peptide_ids = []
id_file.load(filename_in, protein_ids, peptide_ids)
# process
id_file.store(filename_out, protein_ids, peptide_ids)
\end{verbatim}

\texttt{PepXML} also supports loading with an additional parameter specificing
an MSExperiment which contains the retention time corresponding to the peptide
hits (since these may not be stored in a pep.xml file).

see also Section~\ref{PepXMLFile}.

\subsection{ProtXML}
\begin{verbatim}
from pyopenms import *
id_file = ProtXMLFile()
filename_in = "input.prot.xml"
protein_id = ProteinIdentification()
peptide_id = PeptideIdentification()
id_file.load(filename_in, protein_id, peptide_id)
# process
# storing not supported
\end{verbatim}

\texttt{ProtXML} currently only supports loading of data.

\subsection{MzIdentML}
\begin{verbatim}
from pyopenms import *
id_file = MzIdentMLFile()
filename_in = "input.mzid"
filename_out = "output.mzid"
identification = Identification()
id_file.load(filename_in, identification)
# process
id_file.store(filename_out, identification)
\end{verbatim}

Alternatively, MzIdentMLFile also provides a function to load (but not store)
data equivalent to IdXML using two empty vectors that will be filled with
\texttt{ProteinIdentification} and \texttt{PeptideIdentification} objects.

\subsection{TraML}
\begin{verbatim}
from pyopenms import *
tramlfile = TraMLFile()
filename_in = "input.TraML"
targeted_exp = TargetedExperiment()
idtramlfile.load(filename_in, targeted_exp)
# process
tramlfilefile.store(filename_out, targeted_exp)
\end{verbatim}

see also Section~\ref{TraMLFile}.

\subsection{MzML}
\begin{verbatim}
from pyopenms import *
file = MzMLFile()
exp = MSExperiment()
filename_in = "input.mzML"
filename_out = "output.mzML"
file.load(filename_in, exp)
# process
file.store(filename_in, exp)
\end{verbatim}

see also Section~\ref{MzMLFile}.

\pagebreak
\section{Parameter Handling}

Paramter handling in OpenMS and pyOpenMS is usually implemented through
inheritance from \texttt{DefaultParamHandler} and allow access to parameters
through the \texttt{Param} object. This means, the classes implement the
methods \texttt{getDefaults}, \texttt{getParameters}, \texttt{setParameters}
which allows access to the default parameters, the current parameters and
allows to set the parameters.

The Param file that is returned can be manipulated through the \texttt{setValue} and
\texttt{getValue} methods (the \texttt{existsts} method can be used to check for existance of a
key). Using the \texttt{getDescription} method, it is possible to get a help-text for
each parameter value in an interactive session without consulting the documentation.

\section{Signal Processing and Filter}
Most signal processing algorithms follow a similar pattern in OpenMS.

\begin{verbatim}
filter = FilterObject()
exp = MSExperiment()
# populate exp
filter.filterExperiment(exp)
\end{verbatim}

Since they work on a single \texttt{MSExperiment} object, little input is needed to
execute a filter directly on the data. Examples of filters that follow this
pattern are \texttt{GaussFilter, SavitzkyGolayFilter} as well as the spectral
filters \texttt{BernNorm, MarkerMower, NLargest, Normalizer, ParentPeakMower,
Scaler, SpectraMerger, SqrtMower, ThresholdMower, WindowMower}.

There are multiple ways to access the raw data from mzMl files

\begin{verbatim}
from pyopenms import *
file = MzMLFile()
exp = MSExperiment()
filename_in = "input.mzML"
file.load(filename_in, exp)
for spec in exp:
  for peak in spec:
    print peak

# alternatively 
for spec in exp:
  peaks = spec.get_peaks()
  # process peaks
\end{verbatim}

\section{Complex algorithmic tools}

More complex algorithmic tools require a short explanation and usage:

\subsection{Centroided FeatureFinder}

The FeatureFinder for centroided data is called
\texttt{FeatureFinderAlgorithmPicked} in OpenMS.

\begin{verbatim}
# set input_path and out_path
seeds = FeatureMap()

fh = MzMLFile()
options = PeakFileOptions()
options.setMSLevels([1,1])
fh.setOptions(options)
input_map = MSExperiment()
fh.load(input_path, input_map)
input_map.updateRanges()

ff = FeatureFinder()
ff.setLogType(LogType.CMD)

# Run the feature finder
features = FeatureMap()
name = FeatureFinderAlgorithmPicked.getProductName()
params = FeatureFinder().getParameters(name)
ff.run(name, input_map, features, params, seeds)

features.setUniqueIds()
fh = FeatureXMLFile()
fh.store(out_path, features)
\end{verbatim}

\subsection{OpenSwathAnalyzer}

The OpenSwathAnalyzer calls internally an object called
MRMFeatureFinderScoring (since it does feature finding based on a scoring
approach). It takes as input the chromatograms and the targeted library
(transition library) in TraML format. Furthermore, it also takes a
transformation description and a Swath file as optional arguments.

\begin{verbatim}
# load chromatograms
chromatograms = pyopenms.MSExperiment()
fh = pyopenms.FileHandler()
fh.loadExperiment("infile.mzML", chromatograms)

# load TraML file
targeted = pyopenms.TargetedExperiment();
tramlfile = pyopenms.TraMLFile();
tramlfile.load("tramlfile.TraML", targeted);

# Create empty files as input and finally as output
empty_swath = pyopenms.MSExperiment()
trafo = pyopenms.TransformationDescription()
output = pyopenms.FeatureMap();

# set up OpenSwath analyzer (featurefinder) and run
featurefinder = pyopenms.MRMFeatureFinderScoring()
featurefinder.pickExperiment(chromatograms, output, targeted, trafo, empty_swath)

# Store outfile
featurexml = pyopenms.FeatureXMLFile()
featurexml.store("outfile.featureXML", output)
\end{verbatim}

\section{Iterators}

Several core-OpenMS objects have been adapted to allow iteration in a native
way in Python. These objects are currently \texttt{ConsensusMap, FeatureMap,
MSExperiment, MSSpectrum} and \texttt{MSChromatogram}. They thus allow the
following syntax:

\begin{verbatim}
for spectrum in ms_experiment:
  for peak in spectrum:
    # process an individual peak
    pass
\end{verbatim}

\section{Appendix}
In this appendix, a complete list of all wrapped functions is given, ordered
by class. Note that not all C++ functions are wrapped in Python and the
following documentation indicates which do have wrappers and can be used
directly from Python. The following appendix also does not contain actual
documentation of the functionality, please use the link to the OpenMS
documentation to find up-to-date documentation on each class and member
function.


There is nothing here yet, please run \texttt{create\_manual.py} first.



\end{document}
